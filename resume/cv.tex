% Options for packages loaded elsewhere
\PassOptionsToPackage{unicode}{hyperref}
\PassOptionsToPackage{hyphens}{url}
\documentclass[
]{article}
\usepackage{xcolor}
\usepackage[margin=1in]{geometry}
\usepackage{amsmath,amssymb}
\setcounter{secnumdepth}{-\maxdimen} % remove section numbering
\usepackage{iftex}
\ifPDFTeX
  \usepackage[T1]{fontenc}
  \usepackage[utf8]{inputenc}
  \usepackage{textcomp} % provide euro and other symbols
\else % if luatex or xetex
  \usepackage{unicode-math} % this also loads fontspec
  \defaultfontfeatures{Scale=MatchLowercase}
  \defaultfontfeatures[\rmfamily]{Ligatures=TeX,Scale=1}
\fi
\usepackage{lmodern}
\ifPDFTeX\else
  % xetex/luatex font selection
\fi
% Use upquote if available, for straight quotes in verbatim environments
\IfFileExists{upquote.sty}{\usepackage{upquote}}{}
\IfFileExists{microtype.sty}{% use microtype if available
  \usepackage[]{microtype}
  \UseMicrotypeSet[protrusion]{basicmath} % disable protrusion for tt fonts
}{}
\makeatletter
\@ifundefined{KOMAClassName}{% if non-KOMA class
  \IfFileExists{parskip.sty}{%
    \usepackage{parskip}
  }{% else
    \setlength{\parindent}{0pt}
    \setlength{\parskip}{6pt plus 2pt minus 1pt}}
}{% if KOMA class
  \KOMAoptions{parskip=half}}
\makeatother
\usepackage{graphicx}
\makeatletter
\newsavebox\pandoc@box
\newcommand*\pandocbounded[1]{% scales image to fit in text height/width
  \sbox\pandoc@box{#1}%
  \Gscale@div\@tempa{\textheight}{\dimexpr\ht\pandoc@box+\dp\pandoc@box\relax}%
  \Gscale@div\@tempb{\linewidth}{\wd\pandoc@box}%
  \ifdim\@tempb\p@<\@tempa\p@\let\@tempa\@tempb\fi% select the smaller of both
  \ifdim\@tempa\p@<\p@\scalebox{\@tempa}{\usebox\pandoc@box}%
  \else\usebox{\pandoc@box}%
  \fi%
}
% Set default figure placement to htbp
\def\fps@figure{htbp}
\makeatother
\setlength{\emergencystretch}{3em} % prevent overfull lines
\providecommand{\tightlist}{%
  \setlength{\itemsep}{0pt}\setlength{\parskip}{0pt}}
\usepackage{xcolor}
\usepackage{hyperref}
\definecolor{accentcolor}{RGB}{201,114,72}
\hypersetup{colorlinks=true, linkcolor=accentcolor, urlcolor=accentcolor, citecolor=accentcolor}
\usepackage{titlesec}
\titleformat{\section}{\Large\bfseries\color{black}}{\thesection}{1em}{}[\titlerule]
\titlespacing{\section}{0pt}{12pt}{6pt}
\usepackage{bookmark}
\IfFileExists{xurl.sty}{\usepackage{xurl}}{} % add URL line breaks if available
\urlstyle{same}
\hypersetup{
  pdftitle={Dr.~Alexander Shakeel Bates},
  hidelinks,
  pdfcreator={LaTeX via pandoc}}

\title{Dr.~Alexander Shakeel Bates}
\author{}
\date{\vspace{-2.5em}}

\begin{document}
\maketitle

\section[\href{https://as-bates.netlify.app/portfolio/}{Dr.~Alexander
Shakeel Bates}
]{\texorpdfstring{\href{https://as-bates.netlify.app/portfolio/}{Dr.~Alexander
Shakeel Bates}
\protect\includegraphics[width=0.83333in,height=\textheight,keepaspectratio]{asb_logo_round.png}}{Dr.~Alexander Shakeel Bates }}\label{dr.-alexander-shakeel-bates}

\textbf{Neuroscientist \& Computational Biologist}

I am a neuroscientist and computational biologist specialising in
neuroanatomy, neurophysiology and connectomics of the insect brain. My
research focuses on understanding how neural circuits wire and fire to
generate complex behaviours, including olfactory processing and animal
navigation. I develop open-source tools for neuroanatomical analysis and
collaborate internationally on connectomics projects. In the wetlab, I
use virtual reality and calcium imaging experiments to interrogate
neurobiological circuits in living, behaving flies. I am a UK citizen.

\begin{center}\rule{0.5\linewidth}{0.5pt}\end{center}

\textbf{Contact:}
\href{mailto:alexander_bates@hms.harvard.edu}{\nolinkurl{alexander\_bates@hms.harvard.edu}}
\textbar{} \href{https://orcid.org/0000-0002-1195-0445}{ORCID:
0000-0002-1195-0445} \textbar{}
\href{https://github.com/alexanderbates}{GitHub: alexanderbates}

\textbf{Metrics:} 4312 citations \textbar{} h-index: 21 \textbar{}
i10-index: 24 \textbar{} 16 peer reviews

\subsection{Professional Research
\_\_\_\_\_\_\_\_\_\_\_\_\_\_\_\_\_\_\_\_\_\_\_\_\_\_\_\_\_\_\_\_\_}\label{professional-research-_________________________________}

\textbf{Postdoctoral Fellow in Neurobiology} • Harvard Medical School •
Boston, US • present - 01/10/2020

Investigating navigational circuitry using calcium imaging,
neurophysiology and behavioural studies involving virtual reality with
\emph{Drosophila melanogaster}, in the laboratory of
\href{https://neuro.hms.harvard.edu/faculty-staff/rachel-wilson}{Prof.~Rachel
Wilson} • Co-leading an international collaboration with
\href{https://www.lee.hms.harvard.edu/}{Prof.~Wei-Chung Allen Lee},
\href{https://codex.flywire.ai/}{flywire} and a network of international
research groups on the first whole fly central nervous system
connectome, open-access

\subsection{Fellowships \& Grants
\_\_\_\_\_\_\_\_\_\_\_\_\_\_\_\_\_\_\_\_\_\_\_\_\_\_\_\_\_\_\_\_\_\_\_}\label{fellowships-grants-___________________________________}

\textbf{Harvard Nomination for The Warren Alpert Distinguished Scholar
Award} • N/A • TBD • Current - 01/10/2022

Decision not yet made

\textbf{\href{https://wellcome.org/grant-funding/schemes/sir-henry-wellcome-postdoctoral-fellowships}{Sir
Henry Wellcome Fellowship}} • Wellcome Trust \& University of Oxford •
UK • 01/02/2026 - 01/06/2022

300,000 GBP towards my current research • Collaboration between groups
of Rachel Wilson, Wei Lee, Scott Waddell and Jan Drugowitsch • Also
accepted as Life Science Research Foundation fellow

\textbf{\href{https://www.embo.org/funding/fellowships-grants-and-career-support/postdoctoral-fellowships/}{EMBO
fellow}} • European Molecular Biology Organization • Europe • 01/06/2022
- 01/04/2021

Also accepted as a International Human Frontier Science Program fellow

\textbf{\href{https://www.herchelsmith.cam.ac.uk/phd-studentships/current-phd-students}{Herchel
Smith PhD Scholarship}} • Herchel Smith Foundation • Cambridge, UK •
30/09/2019 - 01/08/2015

\textbf{\href{https://bifonds.de/fellowships-grants/phd-fellowships.html}{Boehringer
Ingelheim PhD Scholarship}} • Boehringer Ingelheim Foundation • European
• 01/08/2018 - 01/08/2016

\subsection{Education
\_\_\_\_\_\_\_\_\_\_\_\_\_\_\_\_\_\_\_\_\_\_\_\_\_\_\_\_\_\_\_\_\_\_\_\_\_\_\_\_\_\_\_\_\_\_\_}\label{education-_______________________________________________}

\textbf{Neuroscience PhD} • \href{https://www2.mrc-lmb.cam.ac.uk/}{MRC
LMB}, University of Cambridge • Cambridge, UK • 30/02/2021 - 01/09/2015

PhD student with
\href{https://en.wikipedia.org/wiki/Gregory_Jefferis}{Dr.~Greg Jefferis}
•
\href{https://drive.google.com/file/d/1qiiAcquT948EcSC0SJYW85j8C-amOKkT/view}{\textbf{Thesis}}:
The lateral horn, a brain region in the fly, primes innate olfactory
behaviours by combining patterns of second-order olfactory projection
neuron activity. In my work, I developed tools and analyses, and
reconstructed neural networks from electron microscopy data, in order to
better understand this brain region and how memory systems interact with
it • Neuroinformatics, data science, R programming • \textbf{Awards:}
\href{https://www.cambridgetrust.org/scholarships/v-c-awards-and-cambridge-international-scholarships/\#:~:text=The\%20aim\%20of\%20the\%20Vice,research\%20leading\%20to\%20a\%20PhD.}{Honorary
Vice Chancellor's Award},
\href{https://www2.mrc-lmb.cam.ac.uk/achievements/lmb-student-prize/}{MRC
LMB Max Perutz Prize 2019}, Winner of the
\href{https://www.bna.org.uk/members/student-prizes/}{British
Neuroscience Association Postgraduate Prize 2020}

\textbf{Neuroscience BSc} • University College London • London, UK •
01/07/2015 - 01/09/2012

1st class degree with honours • Modules taken listed on
\href{https://www.linkedin.com/in/alex-bates-22a265a7/}{linkedIn} •
\textbf{Awards:}
\href{https://www.ucl.ac.uk/biosciences/news/2020/jun/professor-geoffrey-burnstock-1929-2020}{Burnstock
Sessional Prize in Neuroscience BSc} (ranked first in year) (2012--2013)
(2013-2014) (2014-2015), Dean's list for the Faculty of Life Sciences
(2013-2014) (2014-2015),
\href{https://www.physoc.org/grants-and-prizes/prizes/the-rob-clarke-awards/}{Rob
Clarke Award} from the Society of Physiology

\textbf{High School} • Woodbridge High School • London, UK • 01/09/2012
- 01/09/2008

6 A*s at A-level, comprising: Physics, Chemistry, Mathematics, English
Literature, Philosophy and Russian, and in a history related EPQ (level
3) project • 13 A*s at GCSE: English Literature, English Language,
Mathematics, Statistics, Core Science, Additional Science, History,
Philosophy, Geography, French, Italian, Russian and Expressive Arts.
\href{https://www.jackpetcheyfoundation.org.uk/opportunities/grant-programmes/achievement-awards/}{Jack
Petchey Achievement Award}

\subsection{Presentations
\_\_\_\_\_\_\_\_\_\_\_\_\_\_\_\_\_\_\_\_\_\_\_\_\_\_\_\_\_\_\_\_\_\_\_\_\_\_\_\_\_\_}\label{presentations-__________________________________________}

\textbf{Selected Talks:} 4th Asia-Pacific Drosophila Neuroscience
Conference (APDNC4) (2026) • CSHL Neurobiology of Drosophila (2025) •
FlyWire Townhall (2025) • HMS Neurobiology Department Talk (2025) • CSHL
Neuronal Circuits (2024) • ECRO meeting (2019) • Boehringer Ingelheim
Meeting (2018) • MPI Connectomics meeting (2017) • ECRO meeting (2017) •
Boehringer Ingelheim Meeting (2017) • Brains and Roses (2016)

\textbf{Selected Posters:} HHMI Investigators' Meeting (2023) • UK
Neural Computation (2019) • Boehringer Ingelheim Fonds communication
workshop (2017) • Maggot Meeting (2016) • High-resolution circuit
reconstruction meeting (2016) • LMB GSA Symposium (2016)

\subsection{Leadership Experience
\_\_\_\_\_\_\_\_\_\_\_\_\_\_\_\_\_\_\_\_\_\_\_\_\_\_\_\_\_\_\_\_\_\_}\label{leadership-experience-__________________________________}

COSYNE 2026 abstract reviewer (2026) • Teaching at the San Juan Winter
School for Connectomics (2025) • Attended HFP Leadership and Management
Skills Course for Postdocs (2023) • President of BlueSci (01/10/2019 -
01/01/2016) • Officially mentored summer student (2018) • Officially
mentored undergraduate student (01/05/2018 - 01/09/2017) • Officially
mentored summer student (2017) • LMB graduate symposium lead organiser
(2017) • LMB graduate symposium organiser (2016) • President of the UCLU
Writer's Society (01/10/2015 - 01/10/2014) • Science Editor, Pi Magazine
(01/10/2015 - 01/10/2014) • UCL iGEM 2014 Advisor (2014)

\subsection{Other Experience
\_\_\_\_\_\_\_\_\_\_\_\_\_\_\_\_\_\_\_\_\_\_\_\_\_\_\_\_\_\_\_\_\_\_\_\_\_\_\_}\label{other-experience-_______________________________________}

\textbf{Research Visits:} Janelia Research Campus (2019) \textbar{}
Worked in FlyEM,
\href{https://www.hhmi.org/scientists/gerald-m-rubin}{Dr.~Gerry Rubin's
Group}, Worked on the hemibrain connectome • Janelia Research Campus
(2016) \textbar{} Worked with
\href{https://www.janelia.org/people/albert-cardona}{Dr.~Albert
Cardona's Group}, Worked on the L1 larval connectome • University of
Queensland Winter Scholarship, University of Queensland (2015)
\textbar{} Worked on tectal activity in zebrafish larvae, light sheet
imaging, \href{https://researchers.uq.edu.au/researcher/1835}{Dr.~Ethan
Scott's Group} •
\href{https://amgenscholars.com/alumni-profile-alex-bates/}{Amgen
Scholarship}, Dept. Zoology, University of Cambridge (2014) \textbar{}
Worked on neuronal structural plasticity in \emph{Drosophila
melanogaster} larvae,
\href{https://www.zoo.cam.ac.uk/directory/dr-matthias-landgraf}{Dr.~Landgraf's
group}\n

\textbf{Training:} Paris Spring School in Neuroscience Techniques, Paris
Descartes University (2018) \textbar{}
\href{https://www.google.com/url?q=https://parisneuro.ovh/&sa=D&source=editors&ust=1626576404921000&usg=AOvVaw3eOj4rTk3Uv6H6IJSM3JZb}{A
course in} Optical Imaging and Electrophysiological Recording in
Neuroscience • UCL iGEM 2013 team member, University College London
(2013) \textbar{} Team member, cloning, cell culture, project planning,
Gold medallist • Summer student in the biomolecular modelling
laboratory, Cancer Research UK, London Research Institute (2013)
\textbar{} Student Placement with
\href{https://bmm.crick.ac.uk/~cheng03/Biography.html}{Dr.~Tammy Cheng},
python programming\n

\textbf{Technical Skills:} R • python • MATLAB • github • git • markdown
• Illustrator • InDesign • communication • text editing • journalistic
writing • creative writing • open access

\subsection{Data and Code
\_\_\_\_\_\_\_\_\_\_\_\_\_\_\_\_\_\_\_\_\_\_\_\_\_\_\_\_\_\_\_\_\_\_\_\_\_\_\_\_\_\_\_}\label{data-and-code-___________________________________________}

\textbf{Public Datasets:}\\
-
\href{https://dataverse.harvard.edu/dataset.xhtml?persistentId=doi:10.7910/DVN/8TFGGB}{BANC
Adult Fly Brain Connectome} (DOI: 10.7910/DVN/8TFGGB) - Complete
synaptic-resolution connectome of an adult female Drosophila
melanogaster brain and ventral nerve cord

\textbf{Open Source Software Contributions:}\\
\emph{natverse} (20 repositories):
\href{https://github.com/natverse/nat}{nat} •
\href{https://github.com/natverse/nat.nblast}{nat.nblast} •
\href{https://github.com/natverse/fafbseg}{fafbseg} •
\href{https://github.com/natverse/natverse}{natverse} •
\href{https://github.com/natverse/nat.examples}{nat.examples} •
\href{https://github.com/natverse/rcatmaid}{rcatmaid} •
\href{https://github.com/natverse/mouselightr}{mouselightr} •
\href{https://github.com/natverse/elmr}{elmr} •
\href{https://github.com/natverse/hemibrainr}{hemibrainr} •
\href{https://github.com/natverse/flycircuit}{flycircuit} •
\href{https://github.com/natverse/neuprintr}{neuprintr} •
\href{https://github.com/natverse/nat.ggplot}{nat.ggplot} •
\href{https://github.com/natverse/fishatlas}{fishatlas} •
\href{https://github.com/natverse/nat.h5reg}{nat.h5reg} •
\href{https://github.com/natverse/neuromorphr}{neuromorphr} •
\href{https://github.com/natverse/neuronbridger}{neuronbridger} •
\href{https://github.com/natverse/influencer}{influencer} •
\href{https://github.com/natverse/drvid}{drvid} •
\href{https://github.com/natverse/natverse_hugo}{natverse\_hugo} •
\href{https://github.com/natverse/insectbrainr}{insectbrainr}\\
\emph{wilson-lab} (3 repositories):
\href{https://github.com/wilson-lab/design-files}{design-files} •
\href{https://github.com/wilson-lab/nat-tech}{nat-tech} •
\href{https://github.com/wilson-lab/panels-matlab}{panels-matlab}\\
\emph{htem} (1 repositories):
\href{https://github.com/htem/BANC-project}{BANC-project}\\
\emph{flyconnectome} (4 repositories):
\href{https://github.com/flyconnectome/2020hemibrain_examples}{2020hemibrain\_examples}
• \href{https://github.com/flyconnectome/bancr}{bancr} •
\href{https://github.com/flyconnectome/hemibrain_olf_data}{hemibrain\_olf\_data}
• \href{https://github.com/flyconnectome/crantr}{crantr}

\textbf{Zenodo Datasets:}\\
- \href{https://zenodo.org/records/10593546}{Supplemental Files for
Eckstein and Bates et al., Cell (2024)} (DOI: 10.5281/zenodo.10593546) -
synister\_fw\_mat571\_t11\_synapses.feather - synapse level transmitter
predictions for the FAFB
dataset========================================================================================
Columns:id\&\ldots{}

\subsection{Publications
\_\_\_\_\_\_\_\_\_\_\_\_\_\_\_\_\_\_\_\_\_\_\_\_\_\_\_\_\_\_\_\_\_\_}\label{publications-__________________________________}

\textbf{Distributed control circuits across a brain-and-cord connectome}
(2025). \textbf{\emph{AS Bates}}†‡, JS Phelps†‡, M Kim†, HH Yang†, A
Matsliah, Z Ajabi, E Perlman, \ldots{} \emph{bioRxiv, }in review at
Nature** {[}citations: 11{]}\\
\textbf{Neurotransmitter classification from electron microscopy images
at synaptic sites in Drosophila melanogaster} (2024). N Eckstein†,
\textbf{\emph{AS Bates}}†, A Champion, M Du, Y Yin, P Schlegel, AKY Lu,
\ldots{} \emph{Cell} {[}citations: 205{]}\\
\textbf{Information flow, cell types and stereotypy in a full olfactory
connectome} (2021). P Schlegel†, \textbf{\emph{AS Bates}}†, T Stürner,
SR Jagannathan, N Drummond, J Hsu, \ldots{} \emph{Elife} {[}citations:
157{]}\\
\textbf{Complete connectomic reconstruction of olfactory projection
neurons in the fly brain} (2020). \textbf{\emph{AS Bates}}†, P
Schlegel†, RJV Roberts, N Drummond, IFM Tamimi, \ldots{} \emph{Curr.
Biology} {[}citations: 207{]}\\
\textbf{The natverse, a versatile toolbox for combining and analysing
neuroanatomical data} (2020). \textbf{\emph{AS Bates}}†, JD Manton†, SR
Jagannathan, M Costa, P Schlegel, T Rohlfing, \ldots{} \emph{Elife}
{[}citations: 206{]}\\
\textbf{Analysis and optimization of equitable US cancer clinical trial
center access by travel time} (2024). H Lee†, \textbf{AS Bates}, S
Callier, M Chan, N Chambwe, A Marshall, MB Terry, \ldots{} \emph{JAMA
oncology} {[}citations: 14{]}\\
\textbf{Functional and anatomical specificity in a higher olfactory
centre} (2019). S Frechter†, \textbf{AS Bates}, S Tootoonian, MJ Dolan,
J Manton, AR Jamasb, \ldots{} \emph{eLife} {[}citations: 115{]}\\
\textbf{Neural circuit mechanisms for steering control in walking
Drosophila} (2025). A Rayshubskiy†, SL Holtz, \textbf{AS Bates}, QX
Vanderbeck, LS Capdevila, \ldots{} \emph{ELife} {[}citations: 101{]}\\
\textbf{Quantitative Attributions with Counterfactuals} (2024). DY
Adjavon†, N Eckstein, \textbf{AS Bates}, GSXE Jefferis, J Funke
\emph{bioRxiv}\\
\textbf{Whole-brain annotation and multi-connectome cell typing of
Drosophila} (2024). P Schlegel†, Y Yin, \textbf{AS Bates}, S Dorkenwald,
K Eichler, P Brooks, DS Han, \ldots{} \emph{Nature} {[}citations:
333{]}\\
\textbf{Discriminative attribution from paired images} (2022). N
Eckstein†, H Bukhari, \textbf{AS Bates}, GSXE Jefferis, J Funke
\emph{Euro. Conf. on Computer Vision} {[}citations: 8{]}\\
\textbf{BAcTrace, a tool for retrograde tracing of neuronal circuits in
Drosophila} (2020). S Cachero†, M Gkantia, \textbf{AS Bates}, S
Frechter, L Blackie, A McCarthy, \ldots{} \emph{Nature methods}
{[}citations: 45{]}\\
\textbf{Neurogenetic dissection of the Drosophila lateral horn reveals
major outputs, diverse behavioural functions, and interactions with the
mushroom body} (2019). MJ Dolan†, S Frechter, \textbf{AS Bates}, C Dan,
P Huoviala, RJ Roberts, \ldots{} \emph{Elife} {[}citations: 167{]}\\
\textbf{Communication from learned to innate olfactory processing
centers is required for memory retrieval in Drosophila} (2018). MJ
Dolan†, G Belliart-Guérin, \textbf{AS Bates}, S Frechter, A
Lampin-Saint-Amaux, \ldots{} \emph{Neuron} {[}citations: 117{]}\\
\textbf{Automated reconstruction of a serial-section EM Drosophila brain
with flood-filling networks and local realignment} (2019). PH Li†, LF
Lindsey, M Januszewski, Z Zheng, \textbf{AS Bates}, I Taisz, M Tyka,
\ldots{} \emph{bioRxiv} {[}citations: 105{]}\\
\textbf{Sexual dimorphism in the complete connectome of the Drosophila
male central nervous system} (2025). S Berg†, IR Beckett, M Costa, P
Schlegel, M Januszewski, EC Marin, \ldots{} \emph{bioRxiv} {[}citations:
4{]}\\
\textbf{Comparative connectomics of Drosophila descending and ascending
neurons} (2025). T Stürner†, P Brooks, L Serratosa Capdevila, BJ Morris,
A Javier, S Fang, \ldots{} \emph{Nature} {[}citations: 35{]}\\
\textbf{A Drosophila computational brain model reveals sensorimotor
processing} (2024). PK Shiu†, GR Sterne, N Spiller, R Franconville, A
Sandoval, J Zhou, \ldots{} \emph{Nature} {[}citations: 76{]}\\
\textbf{Network statistics of the whole-brain connectome of Drosophila}
(2024). A Lin†, R Yang, S Dorkenwald, A Matsliah, AR Sterling, P
Schlegel, S Yu, \ldots{} \emph{Nature} {[}citations: 110{]}\\
\textbf{Neuronal wiring diagram of an adult brain} (2024). S
Dorkenwald†, A Matsliah, AR Sterling, P Schlegel, SC Yu, CE McKellar,
\ldots{} \emph{Nature} {[}citations: 479{]}\\
\textbf{The connectome of the adult Drosophila mushroom body provides
insights into function} (2020). F Li†, JW Lindsey, EC Marin, N Otto, M
Dreher, G Dempsey, I Stark, \ldots{} \emph{Elife} {[}citations: 381{]}\\
\textbf{A connectome and analysis of the adult Drosophila central brain}
(2020). LK Scheffer†, CS Xu, M Januszewski, Z Lu, S Takemura, KJ
Hayworth, \ldots{} \emph{eLife} {[}citations: 1113{]}\\
\textbf{Connectomics analysis reveals first-, second-, and third-order
thermosensory and hygrosensory neurons in the adult Drosophila brain}
(2020). EC Marin†, L Büld, M Theiss, T Sarkissian, RJV Roberts, R
Turnbull, \ldots{} \emph{Curr. Biology} {[}citations: 105{]}\\
\textbf{Input connectivity reveals additional heterogeneity of
dopaminergic reinforcement in Drosophila} (2020). N Otto†, MW Pleijzier,
IC Morgan, AJ Edmondson-Stait, KJ Heinz, I Stark, \ldots{} \emph{Curr.
Biology} {[}citations: 80{]}\\
\textbf{Neural circuit basis of aversive odour processing in Drosophila
from sensory input to descending output} (2018). P Huoviala†, MJ Dolan,
FM Love, P Myers, S Frechter, S Namiki, \ldots{} \emph{bioRxiv}
{[}citations: 48{]}\\
\textbf{Combinatorial encoding of odors in the mosquito antennal lobe}
(2023). P Singh†, S Goyal, S Gupta, S Garg, A Tiwari, V Rajput,
\textbf{AS Bates}, \ldots{} \emph{Nature Comm.} {[}citations: 17{]}

† denotes first author\\
‡ denotes corresponding author

\subsection{Review Articles
\_\_\_\_\_\_\_\_\_\_\_\_\_\_\_\_\_\_\_\_\_\_\_\_\_\_\_\_\_\_\_\_\_\_\_\_\_\_\_\_\_}\label{review-articles-_________________________________________}

\scriptsize

\textbf{Systems neuroscience: Auditory processing at synaptic
resolution} (2022). \textbf{\emph{AS Bates}}, G Jefferis \emph{Curr.
Biology} {[}citations: 1{]}\\
\textbf{Neuronal cell types in the fly: single-cell anatomy meets
single-cell genomics} (2019). \textbf{\emph{AS Bates}}, J Janssens, GS
Jefferis, S Aerts \emph{Curr. opinion in neurobiology} {[}citations:
72{]}\\
\normalsize

\subsection{Referees
\_\_\_\_\_\_\_\_\_\_\_\_\_\_\_\_\_\_\_\_\_\_\_\_\_\_\_\_\_\_\_\_\_\_\_\_\_\_\_\_\_\_\_\_\_\_\_}\label{referees-_______________________________________________}

\textbf{Postdoc Supervisor:} Prof.~Rachel Wilson, Harvard Medical
School,
\href{mailto:Rachel_Wilson@hms.harvard.edu}{\nolinkurl{Rachel\_Wilson@hms.harvard.edu}}
\textbf{PhD Supervisor:} Dr.~Gregory Jefferis, MRC Laboratory of
Molecular Biology, Cambridge,
\href{mailto:jefferis@mrc-lmb.cam.ac.uk}{\nolinkurl{jefferis@mrc-lmb.cam.ac.uk}}
\textbf{Key Collaborator:} Prof.~Wei-Chung Allen Lee, Harvard Medical
School,
\href{mailto:Wei-Chung_Lee@hms.harvard.edu}{\nolinkurl{Wei-Chung\_Lee@hms.harvard.edu}}
\textbf{BSc Tutor:} Dr.~Marco Beato, UCL Neuroscience,
\href{mailto:m.beato@ucl.ac.uk}{\nolinkurl{m.beato@ucl.ac.uk}}
\textbf{Supervisee:} Serene Dhawan, Princeton PhD student,
\href{mailto:serenedhawan@gmail.com}{\nolinkurl{serenedhawan@gmail.com}}

\begin{center}\rule{0.5\linewidth}{0.5pt}\end{center}

\emph{Updated: 03 December 2025}

\end{document}
